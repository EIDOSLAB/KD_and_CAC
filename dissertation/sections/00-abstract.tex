\begin{abstract}

Cardiovascular disease (CVD) is the global major cause of death, and predicting them well in advance can be of fundamental importance: in this sense, the presence of calcium lesions in coronary arteries has been shown to be a very good predictor for future CVD events.
Detect and quantify the amount of arterial calcification is currently a semi-automated procedure, called calcium scoring, performed by experts on Computed Tomography (CT) scans.
In the last years many methods based on neural networks have been proposed to perform automatic calcium scoring on CT scans; while this can automate the process, it still requires a CT scan, which is a time-consuming, resource-intensive test that can be invasive for the patient.

On the other hand, chest X-rays (CXR) is an alternative medical imaging technique that allows to visualize the heart and is executed more often than CT scans in routine clinical testing and requires a simpler, more widespread and less expensive device.
Unfortunately calcium scoring is not possible on CXRs and even to detect if calcium is present or not in the cardiac area is an hard task for expert radiologists. In some specific tasks, however, neural networks have been shown to achieve better results than their human counterparts.

This work aims to detect on CXRs if any calcium lesion on coronary arteries is present using neural networks trained on an exceptional dataset composed by CT scans and CXRs of the same patients, labeled by experts with Coronary Artery Calcium (CAC) score.

\end{abstract}